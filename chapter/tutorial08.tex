\chapter{软件缩放}
\label{ch8}
\section{软件缩放库libswscale}
近来ffmpeg添加了新的接口,使用libswscale来处理图像缩放。既然在前面我们使用img_convert来把RGB转换成YUV12,那么我们现在来使用新的接口。新接口更加标准和快速,而且我相信里面有了MMX优化代码。换句话说,它是做缩放更好的方式。

我们将用来缩放的基本函数是sws_scale。但一开始,我们必需建立一个SwsContext的结构体。这将让我们进行想要的转换,然后把它传递给sws_scale函数。类似于在SQL 中的预备阶段或者是在Python中编译的规则表达式regexp。要准备这个上下文,我们使用sws_getContext 函数,它需要我们源的宽度和高度,我们想要的宽度和高度,源的格式和想要转换成的格式,同时还有一些其它的参数和标志。然后我们像使用img_convert一样来使用sws_scale函数,唯一不同的是我们传递给的是SwsContext:
\begin{lstlisting}
#include <ffmpeg/swscale.h> // include the header!

int queue_picture(VideoState *is, AVFrame *pFrame, double pts) {

  static struct SwsContext *img_convert_ctx;
  ...

  if(vp->bmp) {

    SDL_LockYUVOverlay(vp->bmp);

    dst_pix_fmt = PIX_FMT_YUV420P;
    /* point pict at the queue */

    pict.data[0] = vp->bmp->pixels[0];
    pict.data[1] = vp->bmp->pixels[2];
    pict.data[2] = vp->bmp->pixels[1];

    pict.linesize[0] = vp->bmp->pitches[0];
    pict.linesize[1] = vp->bmp->pitches[2];
    pict.linesize[2] = vp->bmp->pitches[1];

    // Convert the image into YUV format that SDL uses
    if(img_convert_ctx == NULL) {
      int w = is->video_st->codec->width;
      int h = is->video_st->codec->height;
      img_convert_ctx = sws_getContext(w, h,
                        is->video_st->codec->pix_fmt,
                        w, h, dst_pix_fmt, SWS_BICUBIC,
                        NULL, NULL, NULL);
      if(img_convert_ctx == NULL) {
    fprintf(stderr, "Cannot initialize the conversion context!\n");
    exit(1);
      }
    }
    sws_scale(img_convert_ctx, pFrame->data,
              pFrame->linesize, 0,
              is->video_st->codec->height,
              pict.data, pict.linesize);
\end{lstlisting}

我们把新的缩放器放到了合适的位置。希望这会让你知道libswscale 能做什么。

就这样,我们做完了!编译我们的播放器:
\begin{lstlisting}
gcc -o tutorial08 tutorial08.c -lavutil -lavformat -lavcodec -lz -lm `sdl-config --cflags --libs`
\end{lstlisting}
享受我们用C写的少于1000行的电影播放器吧。

当然,也还是有很多我们没太关注的东西可以添加进来的。
